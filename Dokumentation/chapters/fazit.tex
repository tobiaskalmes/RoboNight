\chapter{Fazit}
\section{Nichttechnisches Fach}
In der Arbeit für das Fach, hat sich eindeutig herauskristallisiert, warum es sich um ein Nichttechnisches Fach handelt. Das Fachliche war schnell, selbst als Neuling schnell auch dank des Einführungsworkshops schnell erlernt und lies dabei alte Erinnerungen an die Kindheit wieder aufleben, als man noch Stundenlang an seine Kindheitsträume nachgebaut hat.

Die Herausforderungen lagen eindeutig in anderen Bereichen. Während die Teamarbeit locker und unverkrampft von der Bühne ging, gestaltete es sich als schwierig für den unerfahrenen Teil der Gruppe, ohne Erfahrung das Level der Schüler(innen) einzuschätzen. Mangelnde Erfahrung, was man machen kann und was nicht, ließen anfänglich ein wenig Unbehagen aufkommen. Glücklicher Weise gab es genügend alte Aufgaben zur Orientierung und auch der erfahrene Teamkollege konnte dabei helfen. Sobald sich auch nach einigen aufgekommen Unklarheiten seitens des Anforderungsprofils und der Erwartungen an die Ausgearbeiteten Aufgaben geklärt haben, ging die Arbeit leicht von der Hand. 
    
\section{Umgang mit Schülern}
Der Support für die Schüler(innen) war eine besondere Herausforderung. Obwohl man sich in den Aufgabenstellungen gut auskennt und Fragen relativ präzise gestellt wurden, musste man sich schnell auf unterschiedliche Situationen einstellen und schnell mit unterschiedliche Setups wie z.B. NXT-C und NXT-G klarkommen, um sofort den Fehler finden zu können.

Bei der Arbeit mit den Schülern stellte sich heraus, dass das Altersgefälle und der Wissensunterschied keine große Distanz darstellte. Im Gegenteil. Die Schüler kamen mit klaren Problemstellungen auf das Team zu und nach einer Kurzen Einschätzung der Lage konnte man immer hilfreich zur Seite stehen und Teil am Lernerfolg haben.
  
\section{Schwierigkeiten mit der Hard- und Software}
Bei den Ideen für die Workshops hat sich gezeigt, dass nicht alle Lösungen mit Hilfe von NXT-G ohne größere Umstände, die durch Programmfehler verursacht werden, umzusetzen sind. Des weiteren musste man sich mit der Ungenauigkeit der Sensoren abfinden. Diese machen einige Aufgaben nur mit starken Vereinfachungen lösbar. Oft verhinderte die Ungenauigkeit ein das Komparative Element der Aufgaben. Die Ungenauigkeit des Sensors verhinderte einen möglichen Wettbewerb. Gerade im Vorentscheid erwiesen sich die Anfälligkeit von Sensoren auf die Umgebungseinflüsse als Glückspiel für den Erfolg des Roboters. Hier wäre eine höhere Genauigkeit hilfreich.
    
\section{Leistungsgefälle bei Schülern}
Ausgehend von dem Fortschritt der Gruppen beim Lösen der Aufgaben und dem Schwierigkeitsgrad der Aufgaben sollte man in der Zukunft darüber nachdenken, die Workshops um einen Lernteil für die Schüler zu erweitern. Man sollte den Kindern vielleicht ein paar grundlegende Dinge und ein paar Tricks vorweg zeigen. Gerade das können manche Lehrer nicht so gut Abdecken wie Studenten, die den Schülern eine andere Perspektive aufzeigen können und ihnen einen neue Ideen vermitteln können. 
    
\section{Das Fach Robonight}
Das Fach RoboNight hat sich als attraktives Fach für die Studenten präsentiert. Das viel Spaß bei der Arbeit brachte. Wenn gleich am Anfang die Anforderungen an die Studenten nicht richtig klargemacht wurden, konnte dies nachjustiert und durch verlängerte Fristen unkompliziert gelöst werden. Hilfreich wäre in dem Zusammenhang ein Katalog und nicht proprietäre Aufgabenvorlagen, die mit Libre Office/Latex leicht bearbeitet werden können. Gerne würden wir das Fach nochmal machen, wenn wir dürften. 
    
  
    