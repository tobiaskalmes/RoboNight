\chapter{Längenmessung}
\section{Motivation}
In einem Sägewerk ist die Messanlage ausgefallen. Euer Roboter muss jetzt das Messen übernehmen! Da es aufgrund der Maschinen im Sägewerk aus Sicherheitsgründen nicht möglich ist, sich frei zu bewegen, muss die Messung stationär erfolgen.

\begin{capfigure}[Messung Skizze]
	\includegraphics[width=10cm]{images/messen_skizze}
\end{capfigure}

\section{Aufgabenstellung}
In einem Sägewerk ist die Messanlage ausgefallen. Euer Roboter muss jetzt das Messen übernehmen! Da es aufgrund der Maschinen im Sägewerk aus Sicherheitsgründen nicht möglich ist, sich frei zu bewegen, muss die Messung stationär erfolgen. 

\subsection{Lösungshilfe}
Als Hilfe für die Lösung gibt es hier noch zwei Formeln.

\subsubsection{Variante A}
Wenn der Roboter bündig zu einem Ende des Objektes steht, kann man den Satz des Pythagoras verwenden.

$a^2 + b^2 + = c^2$

\begin{capfigure}[Pythagoras]
	\includegraphics[width=6cm]{images/pythagoras.png}
\end{capfigure}

\clearpage
\subsubsection{Variante B}
Wenn der Roboter nicht bündig zum Anfang oder Ende des Objektes steht, dann kann der Kosinussatz verwendet werden.

$a^2 + b^2 - 2ab*cos\gamma = c^2$

\begin{capfigure}[Kosinussatz]
	\includegraphics[width=6cm]{images/kosinus.png}
\end{capfigure}

Zusatzaufgabe: Berechnet die Fläche des Objekts.

\section{Lösung}